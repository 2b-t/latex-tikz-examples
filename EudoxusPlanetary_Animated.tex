% Eudoxus planetary model (animated)
% Author: Tobit Flatscher
%
% Description
% This document creates an animated Eudoxus planetary model by using
% the integrated transformation features and a manual transformation.

% document settings
\documentclass{standalone}
\usepackage{tikz, tikz-3dplot}
\usepackage{xintexpr}
\usetikzlibrary{calc,patterns,fadings,decorations.pathreplacing}
\usepackage{animate}

% useful commands
\newcommand\pgfmathsinandcos[3]{
	\pgfmathsetmacro#1{sin(#3)}
	\pgfmathsetmacro#2{cos(#3)}
}
\newcommand\LongitudePlane[3][current plane]{
	\pgfmathsinandcos\sinEl\cosEl{#2} %elevation
	\pgfmathsinandcos\sint\cost{#3} %azimuth
	\tikzset{#1/.style={cm={\cost,\sint*\sinEl,0,\cosEl,(0,0)}}}
}
\newcommand\LatitudePlane[3][current plane]{
	\pgfmathsinandcos\sinEl\cosEl{#2} %elevation
	\pgfmathsinandcos\sint\cost{#3} %latitude
	\pgfmathsetmacro\yshift{\cosEl*\sint}
	\tikzset{#1/.style={cm={\cost,0,0,\cost*\sinEl,(0,\yshift)}}}
}
\newcommand\DrawLongitudeCircle[2][1]{
	\LongitudePlane{\angEl}{#2}
	\tikzset{current plane/.prefix style={scale=#1}} %angle of "visibility"
	\pgfmathsetmacro\angVis{atan(sin(#2)*cos(\angEl)/sin(\angEl))}
	\draw[current plane] (\angVis:1) arc (\angVis:\angVis+180:1);
}
\newcommand\DrawLatitudeCircle[2][1]{
	\LatitudePlane{\angEl}{#2}
	\tikzset{current plane/.prefix style={scale=#1}}
	\pgfmathsetmacro\sinVis{sin(#2)/cos(#2)*sin(\angEl)/cos(\angEl)} %angle of "visibility"
	\pgfmathsetmacro\angVis{asin(min(1,max(\sinVis,-1)))}
	\draw[current plane] (\angVis:1) arc (\angVis:-\angVis-180:1);
}

% document-wide Tikz options
\tikzset{
	>=latex, %option for nice arrows
	inner sep=0pt,
	outer sep=2pt,
	mark coordinate/.style={inner sep=0pt,outer sep=0pt,minimum size=3pt,
		fill=black,circle}
}

% colours (rgb)
\definecolor{earth}{rgb}{0.3, 0.3, 0.8}
\definecolor{inner}{rgb}{0.44, 0.31, 0.22}
\definecolor{outer}{rgb}{0.55, 0.44, 0.28}

% geometry parameters
\def\offset{80} %sphere radius offset
\def\Ra{2.5} %world sphere radius
\def\Rb{7} %inner sphere radius
\def\Rc{8} %outer sphere radius
\def\Rd{0.75} %sphere planet radius
\def\Re{6} %middle sphere radius
\def\angEl{35} %elevation angle

\begin{document}
	
	\tdplotsetmaincoords{50}{30}
	\begin{tikzpicture}
		\foreach \i in {0,...,359}
		{
			\tdplotsetrotatedcoords{-45}{\i}{110}
			\begin{scope}[tdplot_rotated_coords]
			
			\pgfmathsinandcos\sinC\cosC{\i-\offset}
			\pgfmathsinandcos\sinD\cosD{\i}	
			\coordinate (P\i) at ({0},{\sinC*\Rc},{\cosC*\Rc});
			\end{scope}
		}
	\end{tikzpicture}
	
\begin{animateinline}[loop, poster = first, controls]{24}
\multiframe{90}{iAngle=0+4}{
	
	\tdplotsetmaincoords{50}{30}
	\begin{tikzpicture}
	% bounding box
	\useasboundingbox[tdplot_screen_coords] (-10,-8) rectangle (10,8);
	
	\tdplotsetrotatedcoords{-45}{\iAngle}{110}
	\begin{scope}[tdplot_rotated_coords]
		\pgfmathsinandcos\sinC\cosC{\iAngle-\offset}
		\xintifboolexpr {\iAngle > 180}
		{\draw[outer,line width=0.5mm] (\Rc,0,0) [y={(0,\sinC,\cosC)}] arc(0:180:\Rc);}
		{\draw[outer,line width=0.5mm] (\Rc,0,0) [y={(0,\sinC,\cosC)}] arc(0:-180:\Rc);}	
	
	\end{scope}
	
	
	\tdplotsetrotatedcoords{-45}{\iAngle}{110}
	\begin{scope}[tdplot_rotated_coords]
		\xintifboolexpr {\iAngle > 180}
		{\draw[inner,line width=0.5mm] (\Rb,0,0) arc(0:180:\Rb);}
		{\draw[inner,line width=0.5mm] (\Rb,0,0) arc(0:-180:\Rb);}
	\end{scope}
	
	\draw[black,dashed] (0,-0.85*\Ra,0) -- (0,-\Re,0);
	\filldraw[ball color=earth] (0,0) circle (\Ra);
	\pgfmathsetmacro\a{\iAngle/3-5};
	\pgfmathsetmacro\b{\iAngle/3-35};
	\pgfmathsetmacro\c{\iAngle/3-175};
	\foreach \t in {-80,-60,...,80} { \DrawLatitudeCircle[\Ra]{\t} }
	\foreach \t in {\a,\b,...,\c} { \DrawLongitudeCircle[\Ra]{\t} }
	\draw[black,dashed] (0,0.85*\Ra,0) -- (0,\Re,0);
	\draw[line width=0.5mm] (0,0,0) [y={(0,1,0)}] circle (\Re);
	
	\tdplotsetrotatedcoords{-45}{\iAngle}{110}
	\begin{scope}[tdplot_rotated_coords]
		\xintifboolexpr {\iAngle > 180}
		{\draw[inner,line width=0.5mm] (\Rb,0,0) arc(0:-180:\Rb);}
		{\draw[inner,line width=0.5mm] (\Rb,0,0) arc(0:180:\Rb);}
	\end{scope}
	
	\foreach \i in {1,...,359}
	{
		\pgfmathsetmacro\j{\i-1};
		\draw[line width=0.25mm,gray] (P\j) -- (P\i);
	}
	
	\tdplotsetrotatedcoords{-45}{0}{0}
	\begin{scope}[tdplot_rotated_coords]
		\draw[inner,dashed] (0,\Rb,0) -- (0,\Re,0);
		\draw[inner,dashed] (0,-\Rb,0) -- (0,-\Re,0);
	\end{scope}
	
	\tdplotsetrotatedcoords{-45}{\iAngle}{110}
	\begin{scope}[tdplot_rotated_coords]
		
		\draw[outer,dashed] (\Rb,0,0) -- (\Rc,0,0);
		\draw[outer,dashed] (-\Rb,0,0) -- (-\Rc,0,0);
		
		\pgfmathsinandcos\sinC\cosC{\iAngle-\offset}
		\pgfmathsinandcos\sinD\cosD{\iAngle}
		
		\xintifboolexpr {\iAngle > 180}
		{\draw[outer,line width=0.5mm] (\Rc,0,0) [y={(0,\sinC,\cosC)}] arc(0:-180:\Rc);}
		{\draw[outer,line width=0.5mm] (\Rc,0,0) [y={(0,\sinC,\cosC)}] arc(0:180:\Rc);}
			
		\filldraw[ball color=white] (0, \sinC*\Rc, \cosC*\Rc) [y={(0,\sinD,\cosD)}] circle (\Rd);
		
	\end{scope}
	
	\end{tikzpicture}
}
\end{animateinline}
	
\end{document} 