% Schematic Fourier Analysis
% Author: Tobit Flatscher (https://github.com/2b-t)
%
% Disclaimer
% The following code is based on a previous version of popular community member 
% Jake but I tried to make it more visually appealing. Some noise is added to the 
% curves in order to make them look more realistic.
%
% Description
% This code creates a schematic representation of an FFT analysis. A signal 
% over time is decomposed into its harmonic oscillations with a corresponding 
% frequency using a Fourier series. The amplitude of the frequency components 
% is plotted.


\documentclass[border=10pt]{standalone}

\usepackage{pgfplots}
\pgfplotsset{compat=1.8}
\usepgfplotslibrary{fillbetween} %for fill under curve

\begin{document}
	
	%define colors
	\definecolor{lightgray}{rgb}{0.8, 0.8, 0.8} %grid of coordinate system, axes
	\definecolor{midgray}{rgb}{0.6, 0.6, 0.6} %layer of border
	\definecolor{darkgray}{rgb}{0.4, 0.4, 0.4} %curves, fill under curve
	
	\begin{tikzpicture} %create tikz picture
	\begin{axis}[ %create 3d plot within tikz
	    set layers=standard, %use predefined layers
	    view={50}{30}, %perspective adjustment
	    domain=0:10, %plot limit in time direction
	    samples y=1, %samples for frequency direction
	    unit vector ratio*=1 2 1, %rescale unit vectors
	    hide axis, %do not plot axes
	    xtick=\empty, ytick=\empty, ztick=\empty, %no ticks on coordinate axes
	    clip=false %let me plot outside the coordinate system
	]
		%limit variables
		\def\xmax{100} %limits for curves and layers
		\def\xmin{0}
		\def\ymax{35}
		\def\ymin{5}
		\def\zmax{25}
		\def\zmin{-5}
		\def\xlayer{110} %frequency layer
		\def\sumcurve{0} %sum curve of time signal
		
		%frequency curves
		\pgfplotsinvokeforeach{1,2,3}{ %for each frequency component
			\draw [on layer=background, lightgray] (axis cs:0,#1,0) -- (axis cs:10,#1,0); %axes
			\addplot3 [on layer=main, darkgray, smooth, samples=200] (x,#1,{1.3*sin(2*#1*x*(157))/(#1*2)}); %plot curves (curves are somewhat arbitrary)
		
			\xdef\sumcurve{\sumcurve + sin(#1*x*(157))/(#1*2)} %add current curve to sumcurve
		}
	
	    %transparent layers
		\fill[white,opacity=0.7] (\xmin,0,\zmin) -- (\xmin,0,\zmax) -- (\xmax,0,\zmax) -- (\xmax,0,\zmin) -- cycle; %transparent layer in time space
		\fill[white,opacity=0.7] (\xlayer,\ymin,\zmin) -- (\xlayer,\ymin,\zmax) -- (\xlayer,\ymax,\zmax) -- (\xlayer,\ymax,\zmin) -- cycle; % transparent layer for frequency space
		
		%grid lines
		\pgfplotsinvokeforeach{\xmin,\xmin+5,...,\xmax}{ %create horizontal grid lines (time layer)
			\draw[lightgray,opacity=0.6] (#1,0,\zmin) -- (#1,0,\zmax);
		}
		\pgfplotsinvokeforeach{\ymin,\ymin+2.5,...,\ymax}{ %create horizontal grid lines (frequency layer)
			\draw[lightgray,opacity=0.6] (\xlayer,#1,\zmin) -- (\xlayer,#1,\zmax);
		}
		\pgfplotsinvokeforeach{\zmin,\zmin+5,...,\zmax}{ %create vertical grid lines (both layers)
			\draw[lightgray,opacity=0.6] (\xmin,0,#1) -- (\xmax,0,#1);
			\draw[lightgray,opacity=0.6] (\xlayer,\ymin,#1) -- (\xlayer,\ymax,#1);
		}
	
	    %borders layer
		\draw[midgray] (\xmin,0,\zmin) -- (\xmin,0,\zmax) -- (\xmax,0,\zmax) -- (\xmax,0,\zmin) -- cycle; %time space layer border line
		\draw[midgray] (\xlayer,\ymin,\zmin) -- (\xlayer,\ymin,\zmax) -- (\xlayer,\ymax,\zmax) -- (\xlayer,\ymax,\zmin) -- cycle; %frequency space layer border line
	
	    %sum curve
		\addplot3 [samples=200] (x,0,{\sumcurve+rand/7+0.2*sin(x*400)}); %sum curve time space with added random noise
		
		% frequency curve
		\addplot3 [name path=f,samples=200,domain=0.5:3.5] (11,x,{rand/30+2*sin((x-0.7)*180)^200*e^(-x/2)-0.3}); %experimentally modified curve with noise
		
		%create fill under curve
		\addplot3 [name path=ax,draw=none,samples=2,domain=0.5:3.5] (11,x,-0.5); %create fill axis
		\addplot3 [darkgray] fill between [of=f and ax]; %create fill between curve and axis
		
		%create coordinate axis
		\draw[-{stealth}] (\xmin,0,\zmin-7) -- (\xmax/5.3,0,\zmin-7); %time axis
		\draw[-{stealth}] (\xlayer,\ymin,\zmin-7) -- (\xlayer,\ymin+\ymax/4,\zmin-7); %frequency axis
		
		%axis labels
		\node[scale=0.7] at (0,0,-21) {Time}; %axis time space
		\node[scale=0.7] at (100,22,-32) {Frequency}; %axis frequency space
		
	\end{axis}
	\end{tikzpicture}
	
\end{document}
