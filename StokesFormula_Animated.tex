% Forces for Stokes' formula (animated)
% Author: Tobit Flatscher
%
% Description
% Characteristic measures like drag and lift force can be calculated
% by means of integration of surface stresses. Of particular interest
% is the Stokes' regime (Re << 1) where time is symmetric in space 
% and time as distribution of surface stresses and pressure around 
% a simple sphere may be obtained analytically.
% This vector graphic shows a simple animation of the surface stresses
% along the body contour.

%document settings
\documentclass[english,10pt]{standalone}

\usepackage{verbatim}
\usepackage[ansinew]{inputenc}
\usepackage{color}
\usepackage{tikz}
\usepackage{pgf}
\usepackage{hyperref}
\hypersetup{pdfpagemode=FullScreen}
\usepackage{ifthen}
\usepackage{animate}

% counter for angle position
\newcounter{angle}
\setcounter{angle}{0}

% geometry parameters
\def\rad{2.5} %circle radius
\def\off{0.3} %label offset
\def\deltaangle{5} %the size of the delta angle d0
\def\nodrawangle{20} %maximum angle until which no angle label is plotted
\def\arr{1} %arrow length

% colours (rgb)
\definecolor{lightgray}{rgb}{0.8, 0.8, 0.8}
\definecolor{midgray}{rgb}{0.6, 0.6, 0.6}
\definecolor{darkgray}{rgb}{0.4, 0.4, 0.4}
\definecolor{darkblue}{rgb}{0.5, 0.5, 0.75}
\definecolor{darkgreen}{rgb}{0.5, 0.75, 0.5}

\begin{document}

\begin{animateinline}[loop, poster = first, controls]{30}

% for all angles
\whiledo{\theangle<180}{
    
    \begin{tikzpicture}
    
    % coordinate axes
    \draw[dotted,darkgray] (-3,0)--(3,0) node[below] {};
    \draw[dotted,darkgray] (0,-3)--(0,3) node[left] {};
    \draw[black,thick] (0,0) circle (\rad);
    
    % radius label
	\node[black,below right] at (0,-\rad) {$R$};
	\draw[black,-stealth] (0,-1.2*\rad)--(0,-\rad);
    
    % pre-calculate often used parameters
    \def\acos{cos(\theangle)}
    \def\asin{sin(\theangle)}
    \def\ahcos{cos(\theangle/2)}
    \def\ahsin{sin(\theangle/2)}
    \def\dacos{cos(\theangle+\deltaangle)}
    \def\dasin{sin(\theangle+\deltaangle)}
    
    % area element: list of start and end points
    \node[draw=none] at ({\rad*\acos},  {\rad*\asin})   (A)    {};
    \node[draw=none] at ({\rad*\acos},  {-\rad*\asin})  (B)    {};
    \node[draw=none] at ({\rad*\dacos}, {-\rad*\dasin}) (C)    {};
    \node[draw=none] at ({\rad*\dacos}, {\rad*\dasin})  (D)    {};
    % area element
    \draw[midgray,fill=lightgray] (A.center)--(B.center)--(C.center)--(D.center);
    \node[midgray,right] at ({\rad*\acos},-0.5) {$dA$};
    
    % stresses: list of start and end points
    \node[draw=none] at ({(\rad+\arr)*\acos},     {(\rad+\arr)*\asin})      (E)    {}; %radial stress
    \node[draw=none] at ({(\rad+\arr)*\acos},     {(\rad)*\asin})           (F)    {}; %radial stress (x-component)
    \node[draw=none] at ({\rad*\acos-\arr*\asin}, {\rad*\asin+\arr*\acos})  (G)    {}; %tangential stress
    \node[draw=none] at ({\rad*\acos-\arr*\asin}, {\rad*\asin})             (H)    {}; %tangential stress (x-component)
    
    % current angle
    \draw[red] (0.5,0) arc (0:\theangle:0.5);
    
    % arrows for stresses
    \draw[darkgreen,-stealth] (A.center)--(G.center);
    \draw[darkblue, -stealth] (A.center)--(E.center);
    
    % determined if arrow should be plotted or not
    \pgfmathparse{180-\nodrawangle}
    \ifthenelse{\theangle<\nodrawangle \OR \theangle>\pgfmathresult}{
    	\def\param{-}
    	
    }{
    	\def\param{-stealth}
	}

	% plot tangential stress components
	\draw[darkgreen,densely dotted,\param] (A.center)--(H.center);
	\draw[darkgreen,densely dotted,\param] (H.center)--(G.center);

	% plot radial stress components
	\draw[darkblue,densely dotted,\param]  (A.center)--(F.center);
	\draw[darkblue,densely dotted,\param]  (F.center)--(E.center);
	
	% add labels
	\node[darkblue,draw=none] at ({(\off+\rad+\arr)*\acos}, {(\off+\rad+\arr)*\asin})        (I)    {$\sigma_{rr}$}; %label radial stress
	\node[darkgreen,draw=none] at ({\rad*\acos-(\off+\arr)*\asin}, {\rad*\asin+(\off+\arr)*\acos})  (J)    {$\sigma_{r\Theta}$}; %label tangential stress
	\node[red,draw=none] at ({(\off+\rad)*\dacos}, {(\off+\rad)*\dasin})  (K)    {$d\Theta$}; %angle theta
	
	% determine if label for angle theta should be plotted or not
	\pgfmathparse{180-\nodrawangle}
	\ifthenelse{\theangle>\nodrawangle}{
		\def\param{-stealth}
		\node[red,draw=none] at ({0.7*\ahcos}, {0.7*\ahsin})  (L)    {$\Theta$};
	}{}
    
    % plot slice 0+d0
    \draw[red] (0,0) -- (A.center);
    \draw[red,densely dashed] (0,0) -- (D.center);
    
    % angle counter on left top
    \node[red,right]   at (-3.5,3.5) {$\Theta = \theangle^{\mathrm{o}}$};
    
    % random nodes for field of few only
    \node[black,right] at (-4.5,4) {};
    \node[black,right] at (4.5,-3.1) {};
    \end{tikzpicture}
    
    % advance angle
    \stepcounter{angle}
    \ifthenelse{\theangle<180}{
            \newframe
    }{
            \end{animateinline}
    }
}

\end{document}