% Different scales of fluid modelling
% Author: Tobit Flatscher (https://github.com/2b-t)
%
% Description
% When describing the behaviour of fluids in fluid mechanics 
% one could adapt multiple viewpoints, different scales of 
% modelling. The complicated ordered macroscopic behaviour (where 
% macroscopic state variables such as temperature T, pressure p 
% and velocity v exist) of a fluid also emerges from simple 
% interaction rules between single particles or fictitious particle 
% clusters.
% Therefore one generally distinguishes between three different 
% scales of fluid modelling:
% - Microscopic models: In very rarefied fluids, such as in a 
%   dilute gas, single particles interact with each other through 
%   pairwise elastic collisions that may be described using 
%   classical Newtonian mechanics. The gas is so dilute interactions 
%   between multiple particles at once are rare and a density for a 
%   point in space does not necessarily exist.
% - Macroscopic models: On a macroscopic level fluids are seen as 
%   dense collections of atoms - state variables like density and 
%   pressure must exist. This is the level of the well-known Navier-
%   Stokes equations or the simple Bernoulli equation.
% - Mesoscopic models: As a gas at room temperature contains around
%   10^19 molecules per cm^3, microscopic models for dense fluids 
%   can't be applied with reasonable effort while for dilute gases 
%   macroscopic models do not exist (no density can be found).  
%   Thus probabilistic densities in space as well as in velocity 
%   space (the chance to find a particle in space with a certain
%   microscopic particle velocity) may be introduced. For discrete 
%   models this results in a lattice, a discrete velocity space where
%   the length f the corresponding vector reflects the chance of 
%   finding a particle in the particular space direction.
% The following script creates an image displaying those different 
% modelling scales using random variables and patterns.


%document settings
\documentclass[border=10pt]{standalone}
\usepackage{xintexpr} %for if statements
\usepackage{tikz}
\usetikzlibrary{calc,patterns,decorations.pathmorphing,decorations.markings} %library for patterns

%random number generator
\newcommand{\GenVar}[3]{
	\pgfmathsetmacro{\RandomOne}{random(#1,#2)/#3} %generate two random numbers (x,y coordinates)
	\pgfmathsetmacro{\RandomTwo}{random(#1,#2)/#3}
}


\begin{document}

%create lattice pattern
\tikzset{
lattice/.pic = {
	\fill (0,0) circle [radius=2pt]; %center circle
	\foreach \x in {-1,0,1}{ %for all velocities in x direction
		\foreach \y in {-1,0,1}{ %for al velocities in y direction
			\xintifboolexpr {\x = 0 && \y = 0} %for a velocity with velocity 0 
			{} %do nothing
			{\GenVar{27}{40}{100} %else: generate a new random number
				\draw [-stealth,line width=0.15mm] (\RandomOne*\x,\RandomOne*\y) to (0.08*\x,0.08*\y);}	%plot an arrow with random length
		   }
    }
}
}

%create atom with microscopic velocity
\tikzset{
atom/.pic = {
    \GenVar{2}{14}{10} %generate random number
	\draw [-stealth,line width=0.15mm, color=lightgray] (0,0) to (0.5-\RandomOne,0.5-\RandomTwo); %draw velocity (offset for negative values)
    \fill[color=midgray] (0,0) circle [radius=3pt]; %draw atom
}
}

%define colors:different shades of gray
\definecolor{lightgray}{rgb}{0.8, 0.8, 0.8}
\definecolor{midgray}{rgb}{0.6, 0.6, 0.6}
\definecolor{darkgray}{rgb}{0.4, 0.4, 0.4}

%draw picture
\begin{tikzpicture}[scale=0.75, transform shape] %scale the whole picture
    
    %MESOSCOPIC LEVEL
    \newcount\countx %new counter for x
    \countx=-2 %set x counter value
    \newcount\county %new counter for y
    \county=-2 %set y counter value
    \loop{ %double loop structure
    	\loop
    	\xintifboolexpr {\countx = 0 && \county = 0} %apart from center cell
    	{}
    	{\path[midgray]  (\countx,\county) pic{lattice};} %insert lattice pattern
    	\advance \county 1 %advance y counter
    	\ifnum \county < 3 %check if y loop is over
    	\repeat} %end of y loop
    \advance \countx 1 %advance x counter
    \ifnum \countx < 3 %check if x loop is over
    \repeat %end of x loop
    
    \path[black] (0,0) pic{lattice}; %center lattice pattern
    
    %draw mesoscopic cells
    \countx=-3 %re-adjust counter
    \loop{\draw[fill=none, draw=lightgray] (\countx/2,-2) -- (\countx/2,2);} %vertical lines
    \advance \countx 2
    \ifnum \countx < 4
    \repeat
    
    \county=-3
    \loop{\draw[fill=none, draw=lightgray] (-2,\county/2) -- (2,\county/2);} %horizontal lines
    \advance \county 2
    \ifnum \county < 4
    \repeat

    \draw[fill=none, draw=darkgray] (-0.5,-0.5) rectangle (0.5,0.5); %draw center cell
    
    %MICROSCOPIC LEVEL
    \path[] (-6+1,1.3) pic{atom}; %create atoms at positions with random velocity direction
    \path[] (-6-0.7,-0.7) pic{atom};
    \path[] (-6-1.5,1.6) pic{atom};
    \path[] (-6+1.3,-1) pic{atom};
    \path[] (-6,-1.6) pic{atom};
    \path[] (-6-1.3,0.75) pic{atom};
    \path[] (-6,2) pic{atom};
    \path[] (-6-1.5,-2.5) pic{atom};
    \path[] (-6+1.75,-2.05) pic{atom};
    \path[] (-6-1.95,-1.5) pic{atom};
    \path[] (-6+2.1,2.1) pic{atom};
    \draw [-stealth,line width=0.15mm, color=darkgray] (-6,0.2) to (-6+0.5,0.2-0.5); %colored atom
    \fill[color=black] (-6,0.2) circle [radius=3pt];
    
    
    %MACROSCOPIC LEVEL
    \countx=-1 %re-adjust counters
    \county=-1
    \loop{
    	\loop
    	\xintifboolexpr {\countx = 0 && \county = 0} %apart from center cell
    	{}
    	{\node[draw=none,text=midgray] at (6+2*\countx, 2*\county) {$T, p, \vec v$};} %create label
    	\advance \county 1
    	\ifnum \county < 2
    	\repeat}
    \advance \countx 1
    \ifnum \countx < 2
    \repeat
    
    \node[draw=none,text=black] at (6, 0) {$T, p, \vec v$}; %create center cell label
    \draw[fill=none, draw=lightgray] (6-1,-2) -- (6-1,2); %draw vertical lines
    \draw[fill=none, draw=lightgray] (6+1,-2) -- (6+1,2);
    \draw[fill=none, draw=lightgray] (6-2,1) -- (6+2,1); %draw horizontal lines
    \draw[fill=none, draw=lightgray] (6-2,-1) -- (6+2,-1);
    \draw[fill=none, draw=darkgray] (6-1,-1) rectangle (6+1,1); %draw center cell
    
    
    %overlay excess material with white boxes
    \draw[fill=white, draw=none] (2,-3) rectangle (4,3);
    \draw[fill=white, draw=none] (-4,-3) rectangle (-2,3);
    \draw[fill=white, draw=none] (8,-3) rectangle (8.5,3);
    \draw[fill=white, draw=none] (-8.25,-3) rectangle (-8,3);
    \draw[fill=white, draw=none] (-8,2) rectangle (8,3);
    \draw[fill=white, draw=none] (-8,-3) rectangle (8,-2);
    
    %draw boxes
    \draw[] (-2-6,-2) rectangle (2-6,2);
    \draw[] (-2,-2) rectangle (2,2);
    \draw[] (-2+6,-2) rectangle (2+6,2);
    
    %add labels
    \node[draw=none] at (-6, 2.6) {\LARGE Microscopic}; %add titles
    \node[draw=none] at (0, 2.6) {\LARGE Mesoscopic};
    \node[draw=none] at (6, 2.6) {\LARGE Macroscopic};
    \node[draw=none,align=left] at (-6, -2.9) {\Large Particle collisions \\ Newtonian mechanics}; %add caption
    \node[draw=none,align=left] at (0, -2.9) {\Large Probabilistic approach \\ Fictitious particle clusters};
    \node[draw=none,align=left] at (6, -2.9) {\Large Continuum hypothesis \\ Macroscopic state variables};
    
    %remove excess area
    \pgfresetboundingbox
    \path[use as bounding box, draw=none] (-7.5,-3.55) rectangle (8.5,3);
    

\end{tikzpicture}

\end{document}
